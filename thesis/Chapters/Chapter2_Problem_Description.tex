\chapter{Problem Description}

  In the present chapter the problem is presented, along with an introduction to the previous work. It is structured as follows: Section \ref{ch_2:sect:1} presents the context of the problem and the requirements a replacement should have. Section \ref{ch_2:sect:2} describes deeply the improvements presented and the decisions taken to end in \ref{ch_2:sect:3} with the description of the previous system.

  \section{Requirements} \label{ch_2:sect:1}

    As of the second version of Aerostack, the only localization technique available is based on the recognition of a special type of marker called Aruco, first used for augmented reality applications. It is a fast and reliable technique to estimate the pose of the camera capturing the image. Altough this system works fine for many applications, it imposes the need of preparing the environment, placing this markers in a very precise way and annotating it's exact position before the experiments. While this might not be a problem in an augmented reality like scenario, when it comes to live localization in unknown environment it becomes useless. Hence, a new system for localization is required.

    Along with the aforementioned localization technique comes the navigator which coordinates with a 2D geometric planner to accomplish the mission at hand. As the localization technique is to be changed, leading a new way to perceive the environment, a new navigator and planner will be necessary. 

  \section{Details of the New Features} \label{ch_2:sect:2}

  \section{Previous Work} \label{ch_2:sect:3}

